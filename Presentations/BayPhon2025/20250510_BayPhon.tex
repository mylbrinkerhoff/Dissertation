\documentclass{beamer}

\mode<presentation>
{
  \usetheme{Madrid}
  % or ...
  \setbeamertemplate{bibliography item}{}
  \setbeamercovered{transparent}
  % or whatever (possibly just delete it)
}

\usepackage{fontspec}
\usepackage[english]{babel}
% or whatever
\usepackage{csquotes}
\usepackage[backend=biber,
        style=unified,
        maxcitenames=3,
        maxbibnames=99,
        natbib,
        url=false]{biblatex}
\addbibresource{Dissertation.bib}
\setmainfont{Linux Libertine O}
\setmonofont{CMU Typewriter Text}
\renewcommand{\ttdefault}{cmtt}

% \usepackage[colorlinks,allcolors={black},urlcolor={blue}]{hyperref} %allows for hyperlinks and pdf bookmarks 
\usepackage{graphicx}	%Inserting graphics, pictures, images 		
\usepackage{multicol} %Multicolumn text
\usepackage{multirow} %Useful for combining cells in tablesbrew 
% \usepackage{booktabs} %Enhanced tables
% \usepackage{underscore} %Allows for underscores in text mode
% \usepackage[colorlinks,allcolors={black},urlcolor={blue}]{hyperref} %allows for hyperlinks and pdf bookmarks
\usepackage{url} %allows for urls
\def\UrlBreaks{\do\/\do-} %allows for urls to be broken up
% \usepackage[normalem]{ulem} %strike out text. Handy for syntax
% \usepackage{tcolorbox}
% \usepackage{datetime2}
\usepackage{caption}
\usepackage{subcaption}
\usepackage{langsci-gb4e} % Language Science Press' modification of gb4e
\usepackage{tikz} % Drawing Hasse diagrams
\usetikzlibrary{decorations.pathreplacing}
\usepackage{leipzig} %	Offers support for Leipzig Glossing Rules


\title[Acoustic Landscape] % (optional, use only with long paper titles)
{The Acoustic Landscape of Voice Quality}

% \subtitle{Include Only If Paper Has a Subtitle}

\author[Brinkerhoff] % (optional, use only with lots of authors)
{Mykel Loren Brinkerhoff}
% - Give the names in the same order as the appear in the paper.
% - Use the \inst{?} command only if the authors have different
%   affiliation.

\institute[UC Santa Cruz] % (optional, but mostly needed)
{University of California, Santa Cruz}
% - Use the \inst command only if there are several affiliations.
% - Keep it simple, no one is interested in your street address.

\date[BayPhon 2025] % (optional, should be abbreviation of conference name)
{BayPhon 2025}
% - Either use conference name or its abbreviation.
% - Not really informative to the audience, more for people (including
%   yourself) who are reading the slides online

% \subject{Theoretical Computer Science}
% This is only inserted into the PDF information catalog. Can be left
% out. 



% If you have a file called "university-logo-filename.xxx", where xxx
% is a graphic format that can be processed by latex or pdflatex,
% resp., then you can add a logo as follows:

% \pgfdeclareimage[height=0.5cm]{university-logo}{university-logo-filename}
% \logo{\pgfuseimage{university-logo}}



% Delete this, if you do not want the table of contents to pop up at
% the beginning of each subsection:
% \AtBeginSubsection[]
% {
%   \begin{frame}<beamer>{Outline}
%     \tableofcontents[currentsection,currentsubsection]
%   \end{frame}
% }


% If you wish to uncover everything in a step-wise fashion, uncomment
% the following command: 

%\beamerdefaultoverlayspecification{<+->}


\begin{document}

\begin{frame}
  \titlepage
\end{frame}

% \begin{frame}{Outline}
%   \tableofcontents
%   % You might wish to add the option [pausesections]
% \end{frame}


% Structuring a talk is a difficult task and the following structure
% may not be suitable. Here are some rules that apply for this
% solution: 

% - Exactly two or three sections (other than the summary).
% - At *most* three subsections per section.
% - Talk about 30s to 2min per frame. So there should be between about
%   15 and 30 frames, all told.

% - A conference audience is likely to know very little of what you
%   are going to talk about. So *simplify*!
% - In a 20min talk, getting the main ideas across is hard
%   enough. Leave out details, even if it means being less precise than
%   you think necessary.
% - If you omit details that are vital to the proof/implementation,
%   just say so once. Everybody will be happy with that.
%-----------------------------------------------------------
\section{Motivation}
%-----------------------------------------------------------
%-----------------------------------------------------------
\subsection{The Basic Problem That We Studied}
%-----------------------------------------------------------
\begin{frame}{What is Voice Quality?}
  % - A title should summarize the slide in an understandable fashion
  %   for anyone how does not follow everything on the slide itself.

  \begin{itemize}
  \item
    Describes how the vocal folds vibrate.
  \item
    Used for both paralinguistic \citep{laverVoiceQualityIndexical1968,podesvaStanceWindowLanguageRace2016} and phonological contrasts \citep{espositoCrosslinguisticPatternsPhonation2020}.
  \item Long been established that phonation contrasts have correlates in the acoustic signal (e.g., \citep{fischer-jorgensenPhoneticAnalysisBreathy1968}).
  \end{itemize}
\end{frame}

\begin{frame}{Modeling Voice Quality}
  \begin{itemize}
    \item Long been established that phonation contrasts have correlates in the acoustic signal (e.g., \cite{fischer-jorgensenPhoneticAnalysisBreathy1968,klattAnalysisSynthesisPerception1990}).
    \item \citet{gordonPhonationTypesCrosslinguistic2001} list several types of measures types that can be used:
    \begin{itemize}
    \item<1-> periodicity
    \item<2-> energy
    \item<3-> spectral tilt
    \item<4-> pitch
    \item<5-> duration 
    \end{itemize}
  \item Linguists have used these measure, or combinations of them, to model voice quality in a variety of languages (e.g., \cite{blankenshipTimingNonmodalPhonation2002,brunelleTonePhonationSoutheast2016,espositoAcousticElectroglottographicStudy2012})
  \end{itemize}
\end{frame}

%-----------------------------------------------------------
\subsection{Previous Work}
%-----------------------------------------------------------
\begin{frame}{Make Titles Informative.}
\end{frame}

\begin{frame}{Make Titles Informative.}
\end{frame}


%-----------------------------------------------------------
\section{Our Results/Contribution}
%-----------------------------------------------------------
%-----------------------------------------------------------
\subsection{Main Results}
%-----------------------------------------------------------
\begin{frame}{Make Titles Informative.}
\end{frame}

\begin{frame}{Make Titles Informative.}
\end{frame}

\begin{frame}{Make Titles Informative.}
\end{frame}

%-----------------------------------------------------------
\subsection{Basic Ideas for Proofs/Implementation}
%-----------------------------------------------------------
\begin{frame}{Make Titles Informative.}
\end{frame}

\begin{frame}{Make Titles Informative.}
\end{frame}

\begin{frame}{Make Titles Informative.}
\end{frame}


%-----------------------------------------------------------
\section*{Summary}
%-----------------------------------------------------------
\begin{frame}{Summary}

  % Keep the summary *very short*.
  \begin{itemize}
  \item
    The \alert{first main message} of your talk in one or two lines.
  \item
    The \alert{second main message} of your talk in one or two lines.
  \item
    Perhaps a \alert{third message}, but not more than that.
  \end{itemize}
  
  % The following outlook is optional.
  \vskip0pt plus.5fill
  \begin{itemize}
  \item
    Outlook
    \begin{itemize}
    \item
      Something you haven't solved.
    \item
      Something else you haven't solved.
    \end{itemize}
  \end{itemize}
\end{frame}

\appendix
%-----------------------------------------------------------
\section<presentation>*{\appendixname}
%-----------------------------------------------------------
%-----------------------------------------------------------
\subsection<presentation>*{References}
%-----------------------------------------------------------
\begin{frame}[t,allowframebreaks]
  \frametitle<presentation>{References}
    \printbibliography
\end{frame}

%-----------------------------------------------------------
\subsection<presentation>*{Variable Importance}
%-----------------------------------------------------------
\begin{frame}
  \frametitle<presentation>{Variable Importance}

\end{frame}
\end{document}