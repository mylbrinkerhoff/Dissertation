\begin{abstract}
    This dissertation provides a detailed description and analysis of the Santiago Laxopa Zapotec voice quality system, a minority language spoken by about 1000 people in the municipality of Santiago Laxopa, and how best to capture these patterns acoustically. The language includes four contrastive phonations: modal, breathy, rearticulated, and checked. The last two are types of creaky voice, which are differentiated in terms of their timing and phonetic realizations. 

    This dissertation showcases a framework for analyzing voice quality and which acoustic measures need to be considered in order to capture the phonetic distinctions between the four phonation types. This framework consists of a two-step process of first conducting an multidimensional scaling analysis and determining which acoustic measures contribute the most to the dimensions. This analysis is followed by a random forest analysis to determine which acoustic measures are the most important for distinguishing between the phonation types. By combining these two analyses, we can determine which acoustic measures are the most important for distinguishing between the phonation types and how they are realized in the language by looking where the two analyses overlap. 
    
    Additionally, this dissertation shows that \posscitet{chaiH1H2AcousticMeasure2022} newly proposed acoustic measure residual H1*, which measures the amplitude of the fundamental (H1), captures the differences in Santiago Laxopa Zapotec. Additionally, the results presented in this dissertation suggest that there are several areas that need to be considered when analyzing voice quality. The first two areas of spectral slope and aperiodic voicing/aspiration noise are well understood and established. The novel areas that need to be considered are the amplitude of the fundamental (i.e., residual H1*) and how these phonations are realized across the duration of the vowel. The results presented in this dissertation show that these four areas are crucial for capturing the phonetic distinctions between the four phonation types in Santiago Laxopa Zapotec.
\end{abstract}