%--------------------------------------------------------------------------
\chapter{Testing the laryngeal complexity in SLZ} \label{ch:testing_lc}
%--------------------------------------------------------------------------

%--------------------------------------------------------------------------
\section{What is Laryngeal Complexity?}\label{sec:what_is_lc}
%--------------------------------------------------------------------------

Laryngeal complexity is defined as the contrastive use of tone and phonation on the same syllablic nucleus \citep{silvermanLaryngealComplexityOtomanguean1997,silvermanPhasingRecoverability1997}. This use of both tone and phonation on the same nucleus is one of the defining characteristics of the Oto-Manguean languages \citep{silvermanLaryngealComplexityOtomanguean1997}. However, it is not limited to just these languages. It has also been used to describe the behavior of tone or pitch in languages outside of the Oto-Manguean languages; such as the Tibeto-Burman languages of Mpi and Tamang \citep{silvermanLaryngealComplexityOtomanguean1997,silvermanPhasingRecoverability1997} and the Mayan language Yucatec Mayan \citep{frazierPhoneticsYucatecMaya2013}.
%, and the Germanic language Danish \citep{frazierPhoneticsYucatecMaya2013,penaStodTimingDomain2022,penaProductionPerceptionStod2024}.


%--------------------------------------------------------------------------
\subsection{Phasing and recoverability}\label{sec:phasing_and_recoverability}
%--------------------------------------------------------------------------

According to \citet{silvermanLaryngealComplexityOtomanguean1997,silvermanPhasingRecoverability1997}, one of the defining aspects of laryngeal complexity is the concept of phasing and recoverability. Under this idea, in laryngeally complex vowels the phonation and tone are phased with respect to one another in way that lends itself to a listener's ability to recover the underlying phonation and tone. In practical terms this means that laryngeally complex vowels are composed of two components: a modal voice portion of the vowel where tone is realized, and a non-modal voice portion of the vowel where phonation is realized. For the researcher that means that there are two distinct portions of the vowel that can be analyzed separately and that need to be analyzed temporally rather than spectrally \citep[237]{silvermanLaryngealComplexityOtomanguean1997}.

For example, in the Oto-Manguean language Jalapa Mazatec, breathiness or creakiness is realized only on the first portion of the vowel either as full laryngeal consonant or as a laryngeal feature on the vowel \citep[238]{silvermanLaryngealComplexityOtomanguean1997}. The second portion of the vowel is realized as a modal voice vowel with one of the three tones belonging to the tonemes of the language. This means that the breathiness or creakiness is phased with respect to the tone.

\citeauthor{silvermanLaryngealComplexityOtomanguean1997} argues that there are three principles that help explain why laryngeal complexity needs to be temporally ordered or phased: (i) sufficient acoustic distance, (ii) sufficient articulatory compatibility, and (iii) optimal auditory salience. 

%--------------------------------------------------------------------------
\subsubsection{Sufficient acoustic distance}\label{sec:sufficient_acoustic_distance}
%--------------------------------------------------------------------------
In discussing sufficient acoustic distance Silverman discusses how pitch perception is not reliant only on the fundamental frequency but on the harmonic spacing and the pulse period in the signal. This means that even if the fundamental was obscured or ``masked'' a hearer should be able to perceive pitch. Silverman concludes that when a periodic glottal wave is either obscured or not present (e.g., when there is non-modal phonation), the acoustic signal cannot encode a salient pitch value.

This does make sense if there there is no recoverability of the harmonic or pulse periods. This is what is observed in Mazatec where the harmonic structure is gone and the pulses are indiscernable.
This pattern reminds me of what we see in most aspiration or creakiness cases where there is no recoverable signal. This does stand in contrast to other languages where the harmonic structure and pulse periods are recoverable like in the languages of Mpi and Tamang which he discusses in Section 4. I am curious what is going on in cases where we have things like breathy falsettos and creaky falsettos. In these cases this might have to do with articulatory compatibility between the non-modal phonation and falsetto. This harkens to the sufficient articulatory compatibility, I believe.


%--------------------------------------------------------------------------
\subsubsection{Sufficient articulatory compatibility}\label{sec:sufficient_articulatory_compatibility}
%--------------------------------------------------------------------------

\citet{lindblomEconomySpeechGestures1983}

\citet{humbertConsonantTypesVowel1978}


This model from \citet{gordonPhonationTypesCrosslinguistic2001} was shown in Figure~\ref{fig:phonation_types} and repeated in Figure~\ref{fig:phonation_types_repeat}.

\begin{figure}[h!]
    \centering
    \begin{tikzpicture}
        % Draw the line with arrows at both ends
        \draw[<->, line width=0.5mm] (0,0) -- (10,0);
        
        % Labels underneath the line
        \node[below] at (0,0) {[h]};
        \node[below] at (2,0) {Breathy};
        \node[below] at (5,0) {Modal};
        \node[below] at (8,0) {Creaky};
        \node[below] at (10,0) {[ʔ]};
        
        % Labels above the line
        \node[above] at (0,0) {Open Glottis};
        \node[above] at (10,0) {Closed Glottis};
    \end{tikzpicture}
    \caption{A diagram showing the relationship between breathy, modal, and creaky phonation types. Based on \citet{gordonPhonationTypesCrosslinguistic2001}.}
    \label{fig:phonation_types_repeat}
\end{figure}

%--------------------------------------------------------------------------
\subsubsection{Optimal auditory salience}\label{sec:optimal_auditory_salience}
%--------------------------------------------------------------------------


%--------------------------------------------------------------------------
\subsection{Implicational hierarchy of laryngealization}\label{sec:implicational_hierarchy}
%--------------------------------------------------------------------------



%--------------------------------------------------------------------------
\section{Previous analyses of laryngeal complexity}\label{sec:previous_analyses}
%--------------------------------------------------------------------------


%--------------------------------------------------------------------------
\section{Analysis of laryngeal complexity}\label{sec:analysis_of_lc}
%--------------------------------------------------------------------------


%--------------------------------------------------------------------------
\section{Results}\label{sec:results_of_lc}
%--------------------------------------------------------------------------


%--------------------------------------------------------------------------
\section{Discussion}\label{sec:discussion_of_lc}
%--------------------------------------------------------------------------

%--------------------------------------------------------------------------
\section{Conclusion}\label{sec:conclusion_of_lc}
%--------------------------------------------------------------------------
