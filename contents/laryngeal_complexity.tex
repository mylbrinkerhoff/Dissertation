%--------------------------------------------------------------------------
\chapter{Testing the laryngeal complexity in SLZ} \label{ch:testing_lc}
%--------------------------------------------------------------------------

%--------------------------------------------------------------------------
\section{What is Laryngeal Complexity?}\label{sec:what_is_lc}
%--------------------------------------------------------------------------

Laryngeal complexity is defined as the contrastive use of tone and phonation on the same syllablic nucleus \citep{silvermanLaryngealComplexityOtomanguean1997,silvermanPhasingRecoverability1997}. This use of both tone and phonation on the same nucleus is one of the defining characteristics of the Oto-Manguean languages \citep{silvermanLaryngealComplexityOtomanguean1997}. However, it is not limited to just these languages. It has also been used to describe the behavior of tone or pitch in languages outside of the Oto-Manguean languages; such as the Tibeto-Burman languages of Mpi and Tamang \citep{silvermanLaryngealComplexityOtomanguean1997,silvermanPhasingRecoverability1997} and the Mayan language Yucatec Mayan \citep{frazierPhoneticsYucatecMaya2013}. 
%, and the Germanic language Danish \citep{frazierPhoneticsYucatecMaya2013,penaStodTimingDomain2022,penaProductionPerceptionStod2024}.


%--------------------------------------------------------------------------
\subsection{Phasing and recoverability}\label{sec:phasing_and_recoverability}
%--------------------------------------------------------------------------

According to \citet{silvermanLaryngealComplexityOtomanguean1997,silvermanPhasingRecoverability1997}, one of the defining aspects of laryngeal complexity is the concept of phasing and recoverability. Under this idea, in laryngeally complex vowels the phonation and tone are phased with respect to one another in way that lends itself to a listener's ability to recover the underlying phonation and tone. In practical terms this means that laryngeally complex vowels are composed of two components: a modal voice portion of the vowel where tone is realized, and a non-modal voice portion of the vowel where phonation is realized. For the researcher that means that there are two distinct portions of the vowel that can be analyzed separately and that need to be analyzed temporally rather than spectrally \citep[237]{silvermanLaryngealComplexityOtomanguean1997}.

For example, in the Oto-Manguean language Jalapa Mazatec, breathiness or creakiness is realized only on the first portion of the vowel either as full laryngeal consonant or as a laryngeal feature on the vowel \citep[238]{silvermanLaryngealComplexityOtomanguean1997}. The second portion of the vowel is realized as a modal voice vowel with one of the three tones belonging to the tonemes of the language. This means that the breathiness or creakiness is phased with respect to the tone.

\citeauthor{silvermanLaryngealComplexityOtomanguean1997} argues that there are three principles that help explain why laryngeal complexity needs to be temporally ordered or phased: (i) sufficient acoustic distance, (ii) sufficient articulatory compatibility, and (iii) optimal auditory salience. 

%--------------------------------------------------------------------------
\subsubsection{Sufficient acoustic distance}\label{sec:sufficient_acoustic_distance}
%--------------------------------------------------------------------------

\citet{silvermanLaryngealComplexityOtomanguean1997} argues that sufficient acoustic distance is necessary for the recoverability of the phonation and tone. As Silverman explain, listeners do not rely on the fundamental frequency alone to perceive pitch. Instead, listeners use the harmonic spacing and the pulse period in the signal to perceive pitch \citep{ritsmaFrequenciesDominantPerception1967,remezIntonationSinusoidalSentences1993}. For modal phonation, this means that the harmonic spacing and pulse periods are present and encode a salient pitch value. However, during non-modal phonation, the harmonic spacing and pulse periods are often obscured or not present.

For breathy voice, this means that there is a general weakening 
of the harmonic structure which makes it difficult to recover the pitch by the listener \citep{silvermanPhasingRecoverability1997}. Creaky voice on the other had obscures the pulse periods due to its aperiodic and unstable glottal vibration \citep{ladefogedSoundsWorldLanguages1996}. This is what was observed in Mazatec where the harmonic structure is gone and the pulses are indiscernible \citep{kirkQuantifyingAcousticProperties1993}. Additionally, the perception of pitch is rendered indiscernible when the pulse periods are varied by 10\% or more \citep{rosenbergPitchDiscriminationJittered1966}.

These observations lead \citet{silvermanLaryngealComplexityOtomanguean1997} to conclude that if a period glottal wave is either obscured (as with breathy voice) or not present (as with creaky voice), the acoustic signal cannot encode a salient pitch value. This means that the phonation and tone must be phased with respect to one another in order for the listener to recover the underlying phonation and tone. I will return to this point in Section~\ref{sec:discussion_of_lc}.

%--------------------------------------------------------------------------
\subsubsection{Sufficient articulatory compatibility}\label{sec:sufficient_articulatory_compatibility}
%--------------------------------------------------------------------------

Another important point for the \citeauthor{silvermanLaryngealComplexityOtomanguean1997}'s theory about laryngeal complexity has to deal with the articulatory compatibility of the phonation and tone. One of the guiding ideas behind this principle is that there is a principle of least effort in biological motor systems such as with speech production \citep{lindblomEconomySpeechGestures1983}. According to \citet{lindblomEconomySpeechGestures1983}, speech gestures can be thought of as distinct motor goals in our speech production system. These goals are achieved by the speaker through the coordination of the articulators with the least amount of effort. This means that that the gestures are coordinated in such a way that they are compatible with one another. This manifests itself either through sequencing or coarticulation of the gestures.

A good example of this comes from nasalization. In nasal contexts, the velum is lowered to allow air to pass through the nasal cavity, creating a nasal sound. This velum lowering gesture is compatible with the gestures needed in the oral cavity to produce different vowel qualities. This is what leads to the production of nasal vowels in languages like French or Portuguese. Additionally, this lowering also occurs in languages that do not have contrastive nasal vowels, such as English, where the velum is lowered in anticipation of a nasal consonant. This lowering of the velum is compatible with the gestures needed to produce the vowels in the oral cavity \citep[e.g.,][]{ohalaPhoneticExplanationsNasal1975,chenAcousticCorrelatesEnglish1997,stylerAcousticalPerceptualFeatures2015}.


\citet{humbertConsonantTypesVowel1978}


This model from \citet{gordonPhonationTypesCrosslinguistic2001} was shown in Figure~\ref{fig:phonation_types} and repeated in Figure~\ref{fig:phonation_types_repeat}.

\begin{figure}[h!]
    \centering
    \begin{tikzpicture}
        % Draw the line with arrows at both ends
        \draw[<->, line width=0.5mm] (0,0) -- (10,0);
        
        % Labels underneath the line
        \node[below] at (0,0) {[h]};
        \node[below] at (2,0) {Breathy};
        \node[below] at (5,0) {Modal};
        \node[below] at (8,0) {Creaky};
        \node[below] at (10,0) {[ʔ]};
        
        % Labels above the line
        \node[above] at (0,0) {Open Glottis};
        \node[above] at (10,0) {Closed Glottis};
    \end{tikzpicture}
    \caption{A diagram showing the relationship between breathy, modal, and creaky phonation types. Based on \citet{gordonPhonationTypesCrosslinguistic2001}.}
    \label{fig:phonation_types_repeat}
\end{figure}

%--------------------------------------------------------------------------
\subsubsection{Optimal auditory salience}\label{sec:optimal_auditory_salience}
%--------------------------------------------------------------------------


%--------------------------------------------------------------------------
\subsection{Implicational hierarchy of laryngealization}\label{sec:implicational_hierarchy}
%--------------------------------------------------------------------------



%--------------------------------------------------------------------------
\section{Previous analyses of laryngeal complexity}\label{sec:previous_analyses}
%--------------------------------------------------------------------------



%--------------------------------------------------------------------------
\section{Analysis of laryngeal complexity}\label{sec:analysis_of_lc}
%--------------------------------------------------------------------------


%--------------------------------------------------------------------------
\section{Results}\label{sec:results_of_lc}
%--------------------------------------------------------------------------


%--------------------------------------------------------------------------
\section{Discussion}\label{sec:discussion_of_lc}
%--------------------------------------------------------------------------

%--------------------------------------------------------------------------
\section{Conclusion}\label{sec:conclusion_of_lc}
%--------------------------------------------------------------------------
