%--------------------------------------------------------------------------
\chapter{Testing the laryngeal complexity in SLZ} \label{ch:testing_lc}
%--------------------------------------------------------------------------

%--------------------------------------------------------------------------
\section{What is Laryngeal Complexity?}\label{sec:what_is_lc}
%--------------------------------------------------------------------------

Laryngeal complexity is defined as the contrastive use of tone and phonation on the same syllablic nucleus \citep{silvermanLaryngealComplexityOtomanguean1997,silvermanPhasingRecoverability1997}. This use of both tone and phonation on the same nucleus is one of the defining characteristics of the Oto-Manguean languages \citep{silvermanLaryngealComplexityOtomanguean1997}. However, it is not limited to just these languages. It has also been used to describe the behavior of tone or pitch in languages outside of the Oto-Manguean languages; such as the Tibeto-Burman languages of Mpi and Tamang \citep{silvermanLaryngealComplexityOtomanguean1997,silvermanPhasingRecoverability1997} and the Mayan language Yucatec Mayan \citep{frazierPhoneticsYucatecMaya2013}. 
%, and the Germanic language Danish \citep{frazierPhoneticsYucatecMaya2013,penaStodTimingDomain2022,penaProductionPerceptionStod2024}.


%--------------------------------------------------------------------------
\subsection{Phasing and recoverability}\label{sec:phasing_and_recoverability}
%--------------------------------------------------------------------------

According to \citet{silvermanLaryngealComplexityOtomanguean1997,silvermanPhasingRecoverability1997}, one of the defining aspects of laryngeal complexity is the concept of phasing and recoverability. Under this idea, in laryngeally complex vowels the phonation and tone are phased with respect to one another in way that lends itself to a listener's ability to recover the underlying phonation and tone. In practical terms this means that laryngeally complex vowels are composed of two components: a modal voice portion of the vowel where tone is realized, and a non-modal voice portion of the vowel where phonation is realized. For the researcher that means that there are two distinct portions of the vowel that can be analyzed separately and that need to be analyzed temporally rather than spectrally \citep[237]{silvermanLaryngealComplexityOtomanguean1997}.

For example, in the Oto-Manguean language Jalapa Mazatec, breathiness or creakiness is realized only on the first portion of the vowel either as full laryngeal consonant or as a laryngeal feature on the vowel \citep[238]{silvermanLaryngealComplexityOtomanguean1997}. The second portion of the vowel is realized as a modal voice vowel with one of the three tones belonging to the tonemes of the language. This means that the breathiness or creakiness is phased with respect to the tone.

\citeauthor{silvermanLaryngealComplexityOtomanguean1997} argues that there are three principles that help explain why laryngeal complexity needs to be temporally ordered or phased: (i) sufficient acoustic distance, (ii) sufficient articulatory compatibility, and (iii) optimal auditory salience. 

%--------------------------------------------------------------------------
\subsubsection{Sufficient acoustic distance}\label{sec:sufficient_acoustic_distance}
%--------------------------------------------------------------------------

\citet{silvermanLaryngealComplexityOtomanguean1997} argues that sufficient acoustic distance is necessary for the recoverability of the phonation and tone. As Silverman explain, listeners do not rely on the fundamental frequency alone to perceive pitch. Instead, listeners use the harmonic spacing and the pulse period in the signal to perceive pitch \citep{ritsmaFrequenciesDominantPerception1967,remezIntonationSinusoidalSentences1993}. For modal phonation, this means that the harmonic spacing and pulse periods are present and encode a salient pitch value. However, during non-modal phonation, the harmonic spacing and pulse periods are often obscured or not present.

For breathy voice, this means that there is a general weakening 
of the harmonic structure which makes it difficult to recover the pitch by the listener \citep{silvermanPhasingRecoverability1997}. Creaky voice on the other had obscures the pulse periods due to its aperiodic and unstable glottal vibration \citep{ladefogedSoundsWorldLanguages1996}. This is what was observed in Mazatec where the harmonic structure is gone and the pulses are indiscernible \citep{kirkQuantifyingAcousticProperties1993}. Additionally, the perception of pitch is rendered indiscernible when the pulse periods are varied by 10\% or more \citep{rosenbergPitchDiscriminationJittered1966}.

These observations lead \citet{silvermanLaryngealComplexityOtomanguean1997} to conclude that if a period glottal wave is either obscured (as with breathy voice) or not present (as with creaky voice), the acoustic signal cannot encode a salient pitch value. This means that the phonation and tone must be phased with respect to one another in order for the listener to recover the underlying phonation and tone. I will return to this point in Section~\ref{sec:discussion_of_lc}.

%--------------------------------------------------------------------------
\subsubsection{Sufficient articulatory compatibility}\label{sec:sufficient_articulatory_compatibility}
%--------------------------------------------------------------------------

Another important point for the \citeauthor{silvermanLaryngealComplexityOtomanguean1997}'s theory about laryngeal complexity has to deal with the articulatory compatibility of the phonation and tone. One of the guiding ideas behind this principle is that there is a principle of least effort in biological motor systems such as with speech production \citep{lindblomEconomySpeechGestures1983}. According to \citet{lindblomEconomySpeechGestures1983}, speech gestures can be thought of as distinct motor goals in our speech production system. These goals are achieved by the speaker through the coordination of the articulators with the least amount of effort. This means that that the gestures are coordinated in such a way that they are compatible with one another. This manifests itself either through sequencing or coarticulation of the gestures.

A good example of this comes from nasalization. In nasal contexts, the velum is lowered to allow air to pass through the nasal cavity, creating a nasal sound. This velum lowering gesture is compatible with the gestures needed in the oral cavity to produce different vowel qualities. This is what leads to the production of nasal vowels in languages like French or Portuguese. Additionally, this lowering also occurs in languages that do not have contrastive nasal vowels, such as English, where the velum is lowered in anticipation of a nasal consonant. This lowering of the velum is compatible with the gestures needed to produce the vowels in the oral cavity \citep[e.g.,][]{ohalaPhoneticExplanationsNasal1975,chenAcousticCorrelatesEnglish1997,stylerAcousticalPerceptualFeatures2015}.

According to \citeauthor{silvermanLaryngealComplexityOtomanguean1997}'s theory of laryngeal complexity, this idea of articulatory compatibility is also driving the need to phase phonation and tone. In the case of laryngeal complexity this is because it is assumed that both tone and phonation are produced by the larynx, more specifically the vocal folds and the glottis. This comes from early work on phonation and tone. For tone, \citet{ohalaProductionTone11978} showed that pitch was controlled primarily by the tensing or laxing of the vocal folds which changed the rate at which the vocal folds vibrate. For phonation, it was similarly shown that the amount the vocal folds where held open or closed determined the type of phonation that was produced \citep{ladefogedSoundsWorldLanguages1996}. This aspect of phonation was shown in Figure~\ref{fig:phonation_types} and repeated here as Figure~\ref{fig:phonation_types_repeat}. 

\begin{figure}[h!]
    \centering
    \begin{tikzpicture}
        % Draw the line with arrows at both ends
        \draw[<->, line width=0.5mm] (0,0) -- (10,0);
        
        % Labels underneath the line
        \node[below] at (0,0) {[h]};
        \node[below] at (2,0) {Breathy};
        \node[below] at (5,0) {Modal};
        \node[below] at (8,0) {Creaky};
        \node[below] at (10,0) {[ʔ]};
        
        % Labels above the line
        \node[above] at (0,0) {Open Glottis};
        \node[above] at (10,0) {Closed Glottis};
    \end{tikzpicture}
    \caption{A diagram showing the relationship between breathy, modal, and creaky phonation types. Based on \citet{gordonPhonationTypesCrosslinguistic2001}.}
    \label{fig:phonation_types_repeat}
\end{figure}

For \citeauthor{silvermanLaryngealComplexityOtomanguean1997}'s \citeyear{silvermanLaryngealComplexityOtomanguean1997} theory of laryngeal complexity, the articulatory mechanisms for tone and phonation are exactly the same which leads to a need to phase the two in order to optimally make use of the same articulatory gestures. However, there is a growing body of literature that shows that tone and phonation is much more complex and is reliant on the entire larynx not just the vocal folds (e.g., \cite{eslingVoiceQualityLaryngeal2019}). This matter will be picked up again in Section~\ref{sec:discussion_of_lc} and Chapter~\ref{ch:modeling_lc}.

%--------------------------------------------------------------------------
\subsubsection{Optimal auditory salience}\label{sec:optimal_auditory_salience}
%--------------------------------------------------------------------------


%--------------------------------------------------------------------------
\subsection{Implicational hierarchy of laryngealization}\label{sec:implicational_hierarchy}
%--------------------------------------------------------------------------

Another aspect of \citeauthor{silvermanLaryngealComplexityOtomanguean1997}'s \citeyear{silvermanLaryngealComplexityOtomanguean1997} laryngeal complexity theory is that there is an implicational hierarchy in the phasing and ordering of phonation and tone. This hierarchy is based on how laryngealization appears in three Oto-Manguean languages. In this implicational hierarchy laryngealization can only appear in three ways: prevocalic, postvocalic, or interrupted. In the prevocalic case, the laryngealization appears before the vowel. In the postvocalic case, the laryngealization appears after the vowel. In the interrupted case, the laryngealization interrupts a vowel and appears in the middle.

According to \citet{silvermanLaryngealComplexityOtomanguean1997}, if a language has interrupted laryngealization, it must also have postvocalic laryngealization. If a language has postvocalic laryngealization, it must also have prevocalic laryngealization. In support of his claims \citet{silvermanLaryngealComplexityOtomanguean1997} provides data from three Oto-Manguean languages: Jalapa Mazatec, Comaltepec Chinantec, and Copala Trique. These languages are shown in Table~\ref{tab:implicational_hierarchy}.

\begin{table}[!ht]
    \centering
    \caption{Implicational hierarchy of laryngeal complexity. The symbols h and ʔ represent laryngealization. The symbol V represents where the modal vowel is located in relation to the laryngealization.  Modified from \citet{silvermanLaryngealComplexityOtomanguean1997}.} 
    \label{tab:implicational_hierarchy}
    \begin{tabular}{lccc}
        \lsptoprule
        \textbf{Language} & \textbf{Prevocalic} & \textbf{Postvocalic} & \textbf{Interrupted} \\
        \hline 
        Jalapa Mazatec & hV˥, ʔV˥ & $-$ & $-$ \\
        Comaltepec Chinantec & hV˥, ʔV˥ & Vh˥, Vʔ˥ & $-$ \\
        Copala Trique & hV˥, ʔV˥ & Vh˥, Vʔ˥ & VhV˥, VʔV˥ \\
        \lspbottomrule
    \end{tabular}
\end{table}

For many descripitions of languages with laryngeal complexity, the implicational hierarchy seems to hold. This is certainly the case in the other Trique languages \citep{dicanioPhoneticsPhonologySan2008,dicanioItunyosoTrique2010,dicanioCoarticulationToneGlottal2012,dicanioPhoneticsFortisLenis2012,dicanioCueWeightPerception2014,dicanioGlottalTogglingItunyoso2020,elliottChicahuaxtlaTriqui2016,hollenbachPhonologyMorphologyTone1984}. However, as mentioned by \citet{frazierPhoneticsYucatecMaya2013}, it is not clear how accurate or robust this implicational hierarchy actually is. The reason for this is because it is not always clear if something is a laryngeal consonant or a laryngeal feature on the vowel. Indeed, \citet{silvermanLaryngealComplexityOtomanguean1997,silvermanPhasingRecoverability1997} treats laryngeal consonants and laryngeal features on vowels as the same thing. For example, in many of the Trique languages, the laryngealization is realized as a laryngeal consonant \citep{dicanioPhoneticsPhonologySan2008,dicanioItunyosoTrique2010,dicanioCoarticulationToneGlottal2012,dicanioPhoneticsFortisLenis2012,dicanioCueWeightPerception2014,dicanioGlottalTogglingItunyoso2020,elliottChicahuaxtlaTriqui2016,hollenbachPhonologyMorphologyTone1984}, but in Jalapa Mazatec, the laryngealization is realized as a laryngeal feature on the vowel \citep{kirkQuantifyingAcousticProperties1993,garellekAcousticConsequencesPhonation2011}. 

Another issue with the implicational hierarchy is that it is based on only three languages. This is a rather small sample size and doesn't capture the full range of variation in the Oto-Manguean languages. For example, in many Mixtec languages, laryngealization is understood to be a feature of the vowel rather than a consonant \citep[e.g.,][]{cortesSanSebastianMonte2023,eischensTonePhonationPhonologyPhonetics2022,gerfenPhonologyPhoneticsCoatzospan1999,gerfenProductionPerceptionLaryngealized2005}. Additionally, in many of this languages, the larngealization can appear either in the middle of the vowel, what \citet{silvermanLaryngealComplexityOtomanguean1997} calls interrupted, or at the end of the vowel \citep[e.g.,][]{cortesSanSebastianMonte2023,eischensTonePhonationPhonologyPhonetics2022}. This is directly in contrast to the implicational hierarchy which states that if a language has interrupted laryngealization, it must also have postvocalic laryngealization and prevocalic laryngealization.

Not only is the violation of the implicational hierarchy the case in Mixtecan languages, it is also the case in other branches of the Oto-Manguean languages. It is often the case that Zapotec languages only have interrupted and postvocalic vowels or just interrupted vowels (see \cite{ariza-garciaPhonationTypesTones2018} for a typology of phonation in Zapotec languages). For example, \citet{avelinobecerraTopicsYalalagZapotec2004,avelinoAcousticElectroglottographicAnalyses2010} argues that Yalálag Zapotec only has interrupted laryngealization as a vowel feature and postvocalic laryngealization with a laryngeal consonant. However, there is no prevocalic laryngealization in the language. This is in direct contrast laryngeal complexity's implicational hierarchy. This is also true for laryngealization in Santa Ana del Valle Zapotec \citep{espositoSantaAnaValle2004,espositoAcousticElectroglottographicStudy2012}. This is also true for Santiago Laxopa Zapotec, where there is no prevocalic laryngealization despite having both interrupted and postvocalic laryngealization, see Chapter~\ref{ch:SLZ} for more detailed information.

%--------------------------------------------------------------------------
\section{Previous analyses of laryngeal complexity}\label{sec:previous_analyses}
%--------------------------------------------------------------------------

Previous analyses about laryngeal complexity fall into two categories: (i) descriptive studies and (ii) instrumental studies. In most descriptive studies, the focus has been on describing the patterns for tone and voice quality and how they interact with one another. For example, \citet{frazierPhoneticsYucatecMaya2013} describes the phonetic properties of tone and voice quality in Yucatec Mayan. In this study, \citeauthor{frazierPhoneticsYucatecMaya2013} describes how Yucatec Mayan is one of the few Mayan languages that has developed tonal contrasts. Additionally, it has a series of vowel that have high tone with glottalized that variably surface as either rearticulated vowels\footnote{This is very similar to rearticulated vowels in SLZ, where a vowel has a period of glottalization in the middle of the vowel that is often realized as a glottal stop. This is different from how \citet{bairdPhoneticPhonologicalRealizations2011} describes ``broken'' or ``rearticulated'' vowels in K'ichee' another Mayan language.} or as a vowel with creaky voice. \citeauthor{frazierPhoneticsYucatecMaya2013} notes that for most speakers these vowels show clear evidence of phasing between the tone and the voice quality. In these vowels the first portion of the vowel is always modal and produced with the high tone. The second portion of the vowel is produced with creaky voice which greatly obscures the pitch. 

Not only is this the case in Yucatec Mayan, but it is also the case in languages that primarily have a phonation contrast with tone/pitch as a secondary cue like Danish \citep{fischer-jorgensenPhoneticAnalysisStod1989,gronnumDanishStodLaryngealization2013,penaStodTimingDomain2022,penaProductionPerceptionStod2024}. In Danish, there is a phonation contrast that exists between modal voice and a type of creaky voice which is called stød. Research has shown that stød is also associated with a secondary cue of a heightened \textit{f}0 \citep{fischer-jorgensenPhoneticAnalysisStod1989,gronnumDanishStodLaryngealization2013}. \citet{penaStodTimingDomain2022,penaProductionPerceptionStod2024} showed that even though Danish is not traditionally classified as a laryngeally complex language it still shows evidence of phasing between the primary and secondary cues to stød, with the primary cue of phonation being produced in the second half of the syllablic rhyme and the secondary cue of pitch being produced in the first half of the syllablic rhyme.

In contrast to the descriptive studies, instrumental studies have focused on the acoustic properties of laryngeal complexity, primarily how laryngealization affects \textit{f}0 and the harmonic structure of the vowel. For example, \citet{garellekAcousticConsequencesPhonation2011} found that in Jalapa Mazatec, the laryngealization causes \textit{f}0 perturbation in the first portion of the vowel. \citeauthor{garellekAcousticConsequencesPhonation2011} conclude that this is evidence for the phasing between laryngealization and tone as predicted by \citeauthor{silvermanLaryngealComplexityOtomanguean1997}'s \citeyear{silvermanLaryngealComplexityOtomanguean1997} theory of laryngeal complexity. Additionally, this same phenomenon of \textit{f}0 perturbation has been shown to be the case with laryngealization in some varieties of Trique, again showing that there is phasing between modal and non-modal phonation \citep{dicanioCoarticulationToneGlottal2012}. For these previous studies the main piece of evidence for laryngeal complexity has been the perturbation of the \textit{f}0 signal in the portion of the vowel that is affected by the non-modal phonation.

In more recent studies, researchers have also looked at other measures besides \textit{f}0 to determine if there is phasing. For example \citet{wellerInteractionsToneGlottalization2023,wellerLexicalToneVowel2023,wellerVoiceQualityTone2024} have investigated laryngeal complexity in San Sebastián del Monte Mixtec using a combination of \textit{f}0 and Strength of Excitation (SoE), a measure that correlates to the strength of voicing, measures. They found that there is a clear phasing between modal and non-modal phonation in the rearticulated vowels of the language. 

However, these accounts only offer a limited window into the question of laryngeal complexity. Because there has been a focus on determining whether or not there are \textit{f}0 perturbations in the signal, these previous studies have missed the opportunity to look at the full range of acoustic properties. \citet{wellerInteractionsToneGlottalization2023,wellerLexicalToneVowel2023,wellerVoiceQualityTone2024} have done an excellent job by showing that by looking at SoE, in addition to \textit{f}0, we can get a better understanding of the phasing between tone and voice quality. However, there is still a need to look at other acoustic properties of the signal to get a full understanding of laryngeal complexity. 

The class of harmonic-to-noise ratio measures is one such class that promises to give this better understanding. Harmonic-to-noise ratio measures are a class of measures that look at the ratio of the harmonic energy to the noise energy in the signal. This class of measures has been particularly helpful in determining whether or not there is aperiodicity in the signal \citep{dekromCepstrumBasedTechniqueDetermining1993,ferrerriesgoWhatMakesCepstral2020,garellekPhoneticsVoice2019}. It is well understood that aperiodicity is one of the defining characteristics of non-modal phonation \citep{ladefogedSoundsWorldLanguages1996}. This means that harmonic-to-noise ratio measures can be used to determine if there is aperiodicity in the signal and if there is a clear phasing between modal and non-modal phonation.

The rest of this chapter will focus on analyzing SoE, a measure of the strength of voicing, and two harmonic-to-noise ratio measures, Harmonic-to-Noise Ratio (HNR) < 1500 Hz, which measures the noise in frequency band of 0 Hz to 1500 Hz, and Cepstral Peak Prominence (CPP), a measure of noise across the entire spectrum, in order to determine if there is a clear phasing between modal and non-modal phonation in SLZ using a generalized additive mixed model (GAMM) analysis \citep{hastieGeneralizedAdditiveModels1986,woodGeneralizedAdditiveModels2017,soskuthyGeneralisedAdditiveMixed2017,wielingAnalyzingDynamicPhonetic2018}.

%--------------------------------------------------------------------------
\section{Analysis of laryngeal complexity}\label{sec:analysis_of_lc}
%--------------------------------------------------------------------------


%--------------------------------------------------------------------------
\section{Results}\label{sec:results_of_lc}
%--------------------------------------------------------------------------


%--------------------------------------------------------------------------
\section{Discussion}\label{sec:discussion_of_lc}
%--------------------------------------------------------------------------

\citet{humbertConsonantTypesVowel1978} 


%--------------------------------------------------------------------------
\section{Conclusion}\label{sec:conclusion_of_lc}
%--------------------------------------------------------------------------
