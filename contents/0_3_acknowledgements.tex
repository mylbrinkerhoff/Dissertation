\begin{acknowledgements}
    Acknowledgements in dissertations and theses are always so awkward and sometimes difficult to write. This is because there are no words that can adequately express the gratitude that you feel towards those who have helped you along the way. But I will try my best.
    
    I would like to thank my advisor, Professor Grant McGuire, for his guidance and support throughout my time at the graduate program at the University of California, Santa Cruz. Thanks to him, I have learned and grown as a researcher and as a person. Thanks to his guidance, encouragement, and support, I have been able to complete this dissertation. 

    This dissertation would also not have been possible without the help of my committee members, Professors Jaye Padgett, Ryan Bennett, and Marc Garellek. Their feedback and advice have been invaluable in shaping my research and my writing. Thank you for your time and effort in reading my dissertation and for your thoughtful comments. I am grateful for your support and encouragement. 

    Besides my committee members, I would like to also thank the first linguistic banana slugs I had the pleasure of knowing. Abby and Aaron Kaplan, both alumni from the linguistic program here at UC Santa Cruz, introduced me to the world of phonetics and phonology while I was an undergraduate student at the University of Utah and by their own love of the field, they inspired me to pursue a career in linguistics. I am grateful for their support while an undergraduate and for helping me take the first steps in my career as a linguist.

    Next, I would like to thank Jennifer L. Smith, who was my advisor during my graduate studies at the University of North Carolina at Chapel Hill. Through her guidance and support, I was able to learn how to write and teach linguistics. I am grateful for her support and encouragement during my time at UNC. I would also like to thank the other members of the linguistic community at UNC, including Professors Brian Hsu, Caitlin Smith, Elliott Moreton, Mike Terry, and Paul Roberge for their support and encouragement during my time at UNC.

    Additionally, I would like to thank the members of the Linguistics Department at UC Santa Cruz for their support and encouragement during my time here. I am grateful for the friendships I have made and the support I have received from my fellow students and faculty members. Thank you to Pranav Anand, Adrian Brasoveanu, Maziar Toosarvandani, Junko Ito, Armin Mester, Jess Law, and all the other professors who have made me feel so welcomed.
    
    In particular, I would like to thank my many friends both within and without the linguistics that have supported and helped me. I would like to thank my friends and colleagues, including but not limited to: Maya Wax Cavallaro, Ben Eischens, Andrew Hedding, Max J. Kaplan, Jack Duff, Amy Reynolds, Zach and Jaelynn Horton, Dan Brodkin, Yaqing Cao, Niko Webster, Jonathan Paramore, Cal Boye-Lynn, Hanyoung Byun, and so many more. I am grateful for your support and encouragement. 

    Finally, I would like to thank my wife, Betsy, and my daughter, Maelyn, for their patience and understanding. I could not have done this without them. Without their love and support, I would not have been able to complete this dissertation.

    The text of this dissertation includes reprint of the following previously published material: Mykel Loren Brinkerhoff, Grant McGuire; Using residual H1* for voice quality research. \textit{JASA-EL} 1 February 2025; 5 (2): 025501. \href{https://doi.org/10.1121/10.0035881}{DOI: 10.1121/10.0035881}. The co-author listed in this publication directed and supervised the research which forms the basis for the dissertation.
\end{acknowledgements}

