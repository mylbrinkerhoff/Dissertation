\begin{abstract}
    This dissertation provides a detailed description and analysis of the SLZ voice quality system, a minority language spoken by about 1000 people in the municipality of Santiago Laxopa, and its interactions with the tonal system of the language. Standard assumptions about the interaction between tone and voice quality in Otomanguean languages proposed by \citet{silvermanLaryngealComplexityOtomanguean1997,silvermanPhasingRecoverability1997}, where nonmodal phonation is realized on only a portion of the vowel and tone is realized on a modal portion, do not fully hold in Santiago Laxopa Zapotec. Instead, speakers routinely produce nonmodal phonation throughout the entire vowel for breathy vowels. The only time that phasing is observed is with the two types of creaky voice that occur in the language: rearticulated and checked. Rearticulated vowels have a period of creakiness in the middle of the vowel, whereas checked vowels have creakiness at the end of the vowel. Although this creakiness is pronounced in distinct locations, nonmodal phonation is still present throughout the entire vowel. These results were confirmed through statistical modeling.

\end{abstract}