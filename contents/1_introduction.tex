%--------------------------------------------------------------------------
%  Introduction
%--------------------------------------------------------------------------

%--------------------------------------------------------------------------
\chapter{Introduction} \label{chap:introduction}
%--------------------------------------------------------------------------


%--------------------------------------------------------------------------
\section{What is Voice Quality} \label{sec:voice_quality}
%--------------------------------------------------------------------------
Voice quality describes the state of the larynx during phonation, when the vocal folds are set in motion. Languages make use of voice quality for paralinguistic purposes, such as conveying indexation of ``biological, psychological, and social characteristics of the speaker'' \citep{laverVoiceQualityIndexical1968} and racial identity \citep{podesvaStanceWindowLanguageRace2016}. 

Voice quality is also used linguistically. In English, it is often the case that we use creaky voice to indicate that we are at the end of an utterance \citep[e.g.,][]{garellekProductionPerceptionGlottal2013}. In many other languages, voice quality is used as part of the phonological system. Most famously, Gujarati has a phonemic contrast between breathy and modal voice in vowels \citep[e.g.,][]{fischer-jorgensenPhoneticAnalysisBreathy1968,espositoContrastiveBreathinessConsonants2012, khanPhoneticsContrastivePhonation2012,espositoDistinguishingBreathyConsonants2019}. 

\citet{espositoCrosslinguisticPatternsPhonation2020}

%--------------------------------------------------------------------------
\section{Voice Quality and Tone} \label{sec:voice_quality_and_tone}
%--------------------------------------------------------------------------

%--------------------------------------------------------------------------
\section{Voice Quality and Phonation} \label{sec:voice_quality_and_phonation}
%--------------------------------------------------------------------------

% In other languages, voice quality can be used to distinguish between phonemes, such as in Jalapa Mazatec, where voice quality is used to distinguish between voiced and voiceless stops \citep{merrillVoiceQualityJalapa2011}. Voice quality is also used to convey information about the speaker's emotional state, such as in the case of breathiness, which is often associated with sadness \citep{hentonBreathinessVoiceQuality1996}.

%--------------------------------------------------------------------------
\section{Interactions between Voice Quality and Tone} \label{sec:interactions_between_voice_quality_and_tone}
%--------------------------------------------------------------------------

%------------------------------------
\subsection{Why Esposito \& Khan 2020 matters to my research} \label{sec:WhyEspositoKhanMatters}
%------------------------------------

\begin{itemize}
    \item \citet{espositoCrosslinguisticPatternsPhonation2020} is primarily important for my research because of their typology of the interactions between tone and phonation, which are summarized in Table~\ref{tab:TypologyRepeat}.
\end{itemize}

\begin{table}[h!]
    \centering
    \caption{Language types based on the lexically contrastive nature of \textit{f0}/tone/pitch accents (rows) and voice quality on vowels (columns) and their interactions \citep{espositoCrosslinguisticPatternsPhonation2020}}
    \label{tab:TypologyRepeat}
    \begin{tabular}{c|c|c|c|c}
        \lsptoprule
     & \multicolumn{2}{c|}{\textbf{No VQ contrast}} & \multicolumn{2}{c}{\textbf{VQ contrast}} \\
     \hline
    \textbf{No tone contrast} & \multicolumn{2}{c|}{\begin{tabular}[c]{@{}c@{}}No contrasts \\ (Type I)\end{tabular}} & \begin{tabular}[c]{@{}c@{}}\textit{f0} not a cue \\ (Type IIa)\end{tabular} & \begin{tabular}[c]{@{}c@{}}\textit{f0} is a cue \\ (Type IIb)\end{tabular} \\
    \hline
    \textbf{Tone contrast} & \begin{tabular}[c]{@{}c@{}}VQ not a cue \\ (Type IIIa)\end{tabular} & \begin{tabular}[c]{@{}c@{}}VQ is a cue \\ (Type IIIb)\end{tabular} & \begin{tabular}[c]{@{}c@{}} Orthogonal \\ (Type IVa)\end{tabular} & \begin{tabular}[c]{@{}c@{}} Fused \\ (Type IVb)\end{tabular} \\
    \lspbottomrule
    \end{tabular}
\end{table}

\begin{itemize}
    \item Another aspect of my research into the tone and phonation interactions is why we see gaps/restrictions for which tones and phonations are allowed to combine. 
    \begin{itemize}
        \item This is important because of the gaps I observe in SLZ where some we never see breathy with high tone and we do not see checked vowels with rising tone. 
        \begin{itemize}
            \item This is true for nominals. 
            \item I have not looked at verbs and whether or not we see this same gap. 
            \begin{itemize}
                \item I expect that the gap is also present in the verbal paradigms similar to what \citet{uchiharaToneRegistrogenesisQuiavini2016} observed where breathy phonation fails to appear in some parts of the paradigm.
                \item Especially with the Potential Aspect, which is always realized as a high tone on the verbal root. 
            \end{itemize}
        \end{itemize}
    \end{itemize}
    \item These four types of languages offer an excellent way for me to characterize what we see cross-linguistically in the interactions between tone and phonation. 
    \item This primarily is useful for me as a way to zero in on which languages I need to look at. 
    \begin{itemize}
        \item This means that I need to focus my search into types IIb, IIIb, and IV languages.
    \end{itemize}
\end{itemize}

%------------------------------------
\subsection{Connecting the two} \label{sec:Connection}
%------------------------------------

\begin{itemize}
    \item The connection here between \citet{silvermanLaryngealComplexityOtomanguean1997} and the typology is that these both have to deal with the interactions between tone and phonation. 
    \item Silverman offers an account for what we expect to see when a language has both tone and phonation. 
    \begin{itemize}
        \item It additionally offers three motivating factors for interactions between tone and phonation.
        \item These three factors also offer ways to account for the patterns that we observe and don't observe in the typology
    \end{itemize}
    \item The typology allows us to see more generally what combinations we have and don't have.
    \item If there are any cross-linguistic gaps that we observe, then do \posscitet{silvermanLaryngealComplexityOtomanguean1997} three factors of (i) sufficient acoustic distance, (ii) sufficient articulatory compatibility, and (iii) optimal auditory salience provide the answers to how and why.
\end{itemize}

%------------------------------------
\section{Esposito \& Khan 2020} \label{sec:EspositoKhan}
%------------------------------------

\begin{itemize}
    \item \citet{espositoCrosslinguisticPatternsPhonation2020} is a paper that explains the various cross-linguistic patterns that exist in the world's languages for phonation. 
    \item They describe that for the majority of the world's languages there is only modal voice. 
    \item However, a subset of the languages have non-modal phonation. 
    \item They state that languages can either associate phonation, which they define as production of sound by the vocal folds, with consonants or vowels. 
    \begin{itemize}
        \item For consonants this takes the form of breathy or creaky voice (4ff).
    \end{itemize}
    \item I am going to ignore their discussion of phonation associated with consonants for the most part and focus on what they have to say about vowels. 
    \begin{itemize}
        \item According to them only five languages contrast phonation on both vowels and consonants. 
        \item These are !Xóõ, Ju|'hoansi, Wa, White Hmong, and Gujarati
    \end{itemize}
    \item They describe that there are two main avenues for researching phonation's production. 
    \begin{itemize}
        \item This is done through acoustic measurements and electroglottography
    \end{itemize}
    \item The most common way that this is done is with spectral silt measures which reflect the open quotient.
    \begin{itemize}
        \item The open quotient is the proportion of the glottal cycle during which the glottis is open \citep{holmbergComparisonsAerodynamicElectroglottographic1995}. 
    \end{itemize} 
    \item Other measures that reflect the periodicity of the signal are also consulted such as HNR and CPP. Typically nonmodal phonation is also aperiodic and will have a lower score compared to the modal. 
    \item They also state that ``H1-H2 may be a (near-)universal acoustic measure of phonation'' (8).
    \item They further say that this measure works the best the most often but there are several exceptions where other H1-X measures are more robust at capturing the phonation contrasts in the language. 
    \begin{itemize}
        \item As discussed in \citet{chaiH1H2AcousticMeasure2022}, what they are actually trying to do is get at the amplitude differences in H1 as first observed by \citet{fischer-jorgensenPhoneticAnalysisBreathy1968,laverVoiceQualityIndexical1968}.
        \item I think this is where, I and other researchers fell into a fallicy where we think that just the spectral-tilt is what matters instead of realizing what those measures where trying to accomplish which was to normalize H1 for comparison. 
    \end{itemize}
    \item \citet{espositoCrosslinguisticPatternsPhonation2020} further discuss the role that EGG play. 
    \item Two things that I think are relavent for my dissertation is the discussion around localization of non-modal phonation and how phonation relates to tone. 
    \item For the discussion around localization, they keep in line with \citet{silvermanLaryngealComplexityOtomanguean1997} who says that phonation is located in certain portions of the vowel. 
    \item Localization of non-modal phonation is also important in languages with complex ``clusters'' (i.e., phonetic sequences) of phonation type, which involve at least one non-modal phonation localized to a portion of the vowel. 
    \begin{itemize}
        \item !Xóõ distinguishes clusters of breathy-creaky, pharyngealized-creaky, and breathy-pharyngealized phonation in addition to modal, breathy, creaky, and pharyngealized voice
    \end{itemize}
    \item In terms of tone and phonation interactions, \citet{espositoCrosslinguisticPatternsPhonation2020} claim that there are four different types of interactions based on whether or not tone and phonation is contrastive in the language. 
    \item Table~\ref{tab:Typology} shows each of the different interactions between tone and phonation. In this table each row represents whether or not tone/\textit{f0}/pitch accent is contrastive and each column represents whether or not phonation is contrastive. 
    \item Each cell represent each of the four different types of languages (I-IV). 
    \item Types II-IV are further subdivided based on whether or not tone and phonation are cues for each other. 
\end{itemize}

\begin{table}[h!]
    \centering
    \caption{Language types based on the lexically contrastive nature of \textit{f0}/tone/pitch accents (rows) and voice quality on vowels (columns) and their interactions \citep{espositoCrosslinguisticPatternsPhonation2020}}
    \label{tab:Typology}
    \begin{tabular}{c|c|c|c|c}
        \lsptoprule
     & \multicolumn{2}{c|}{\textbf{No VQ contrast}} & \multicolumn{2}{c}{\textbf{VQ contrast}} \\
     \hline
    \textbf{No tone contrast} & \multicolumn{2}{c|}{\begin{tabular}[c]{@{}c@{}}No contrasts \\ (Type I)\end{tabular}} & \begin{tabular}[c]{@{}c@{}}\textit{f0} not a cue \\ (Type IIa)\end{tabular} & \begin{tabular}[c]{@{}c@{}}\textit{f0} is a cue \\ (Type IIb)\end{tabular} \\
    \hline
    \textbf{Tone contrast} & \begin{tabular}[c]{@{}c@{}}VQ not a cue \\ (Type IIIa)\end{tabular} & \begin{tabular}[c]{@{}c@{}}VQ is a cue \\ (Type IIIb)\end{tabular} & \begin{tabular}[c]{@{}c@{}} Orthogonal \\ (Type IVa)\end{tabular} & \begin{tabular}[c]{@{}c@{}} Fused \\ (Type IVb)\end{tabular} \\
    \lspbottomrule
    \end{tabular}
\end{table}

%------------------------------------
\subsection{Type I languages} \label{sec:TypeI}
%------------------------------------

\begin{itemize}
    \item Type I languages are those languages that lack tonal contrasts and phonation contrasts. 
    \item According to \citet{espositoCrosslinguisticPatternsPhonation2020} these are the languages that belong to this type:
    \begin{itemize}
        \item most Australian Aboriginal languages, 
        \item most Austronesian languages, 
        \item most Indo-European languages
        \item Afro-Asiatic languages, 
        \item Standard Khmer,
        \item Turkic languages
    \end{itemize}
    \item They do not go into much detail about these languages 
\end{itemize}

%------------------------------------
\subsection{Type II languages} \label{sec:TypeII}
%------------------------------------

\begin{itemize}
    \item Type II languages are languages that lack tonal contrasts but do have phonation contrasts on the vowel. 
    \item This is further subdivided into two subtypes (IIa and IIb) based on whether or not changes in \textit{f0} are a cue to the phonation. 
    \item Type IIa languages are those where \textit{f0} is not a cue. 
    \item According to \citet{espositoCrosslinguisticPatternsPhonation2020}, these languages are quite rare and only two languages are known to exhibit this behavior: 
    \begin{itemize}
        \item Danish \citep{gronnumDanishStodLaryngealization2013}
        \item Gujarati \citep{khanPhoneticsContrastivePhonation2012}
    \end{itemize} 
    \item Type IIb languages are those where \textit{f0} is a cue for the different phonation types.
    \item These languages are sometimes called ``register languages''
    \item These are languages such as: 
    \begin{itemize}
        \item Chanthaburi Khmer
        \item Chong 
        \item Javanese
        \item Kedang
        \item Mon
        \item Suai 
        \item Wa
    \end{itemize}
    \item In type IIb languages the way in which \textit{f0} is a cue is language specific. 
    \item They describe that breathy vowels in Wa and Kedang both begin with a lower f0 than their clear/modal counterparts
    \item However, breathy vowels in Chanthaburi Khmer have a higher f0 than their clear counterparts.
\end{itemize}

%------------------------------------
\subsection{Type III languages} \label{sec:TypeIII}
%------------------------------------

\begin{itemize}
    \item Type III languages are characterized by langugaes that have a tonal contrast but lack a phonation contrast. 
    \item These languages are further subdivided into two subtypes based on whether or not phonation is a cue for the tonal contrasts (Type IIIa vs. Type IIIb)
    \item In Type IIIa languages, tone is contrastive and phonation is not a cue for those tonal contrasts.
    \item These are languages like:
    \begin{itemize}
        \item Japanese,
        \item Navajo
        \item Punjabi
        \item Manange
        \item Most W. African languages
        \item Swedish
        \item Central Thai
    \end{itemize}
    \item Type IIIb languages are those that have lexical tones and no lexical vowel phonation. However, subsets of the tone categories have been reported to have optional phonation which helps cue the tonal contrasts. 
    \item For example, tone 3 of Mandarin Chinese which is usually accompanied with creak \citep{kuangCovariationVoiceQuality2017}.
    \item These are languages such as: 
    \begin{itemize}
        \item Cantonese Yue,
        \item Khmu' Rawk
        \item Mandarin
        \item Pakphanang Thai
        \item Ph.Penh Khmer
        \item Yueyang Xiang
    \end{itemize}
\end{itemize}

%------------------------------------
\subsection{Type IV languages} \label{sec:TypeIV}
%------------------------------------

\begin{itemize}
    \item Type IV languages are those languages that have both a contrast in tone and phonation and is also subdivided into two subtypes (IVa and IVb). 
    \item Type IVa are those languages where tone and phonation are completely orthogonal to each other. This means that all tone contrasts can appear with any of the phonation contrasts. 
    \item Type IVa languages include:
    \begin{itemize}
        \item Dinka,
        \item Mazatec language,
        \item Mpi
        \item Yalálag Zapotec
        \item Yi languages such as Bo
    \end{itemize}
    \item Type IVb languages are often called ``register languages'' because tone and phonation are so closely tied to one another that it is impossible to state whether or not they have tonal contrasts or phonation contrasts. 
    \item In these languages suprasegmental categories (i.e., tone) are consistantly produced with both a specific \textit{f0} as well as a specific phonation. 
    \item Examples for this include the behavior where specific tones only arise with specific phonation types. This is most commonly found among languages in Mainland Southest Asia. 
    \item \citet{espositoCrosslinguisticPatternsPhonation2020} additionally describe that most Zapotec languages also fall into this type of language.
    \begin{itemize}
        \item I do not agree with the strong version of this statement. The evidence that they cite for this comes from \citet{espositoVariationContrastivePhonation2010} where she describes that SAV Zapotec syllables fall into one of four categories: breathy phonation with falling tone, creaky phonation with low falling tone, modal phonation with a high tone, and modal phonation with rising tone
        \item It is true that some phonation types are closely tied to one tonal pattern in Zapotec languages. 
        \item For example, in Isthmus Zapotec \citep{pickettIsthmusJuchitanZapotec2010} is described as having five tonal melodies (H, L, LH, HL, and LHL) and three phonation types (modal, checked, and laryngealized). 
        \item In monosyllabic nouns L and LH can surface with any of the three phonation types but the combinations of H and HL with any of the phonation types do not appear to be present in the lexicon
        \item When we consider disyllabic nouns we see the following combinations: 
        \begin{itemize}
            \item L appears with any of the phonation types. 
            \item H appears with modal and laryngealized
            \item LH appears with modal and checked
            \item HL appears with modal and checked
            \item LHL only appears with modal. 
        \end{itemize} 
        \item This is also true in SLZ where the only combination of tone and phonation that we fail to see is H with breathy phonation and MH with checked.
        \item This leads me to believe that maybe it would be better to think of Zapotec languages as a third subtype of languages where some of the tone and phonation contrasts are orthogonal and other tone and phonation contrasts are fused. 
        \item In otherwords a hybrid between Type IVa and Type IVb which I would call mixed languages or Type IVc.
        \item This is illustrated in a new updated table in Table~\ref{tab:TypologyUpdated}.
    \end{itemize}
\end{itemize}

\begin{table}[!h]
    \centering
    \caption{Updated language types based on the lexically contrastive nature of \textit{f0}/tone/pitch accents (rows) and voice quality on vowels (columns) and their interactions}
    \label{tab:TypologyUpdated}
    \begin{tabular}{c|c|c|c|c|c}
        \lsptoprule
     & \multicolumn{2}{c|}{\textbf{No VQ contrast}} & \multicolumn{3}{c}{\textbf{VQ contrast}} \\
     \hline
    \textbf{No tone contrast} & \multicolumn{2}{c|}{\begin{tabular}[c]{@{}c@{}}No contrasts \\ (Type I)\end{tabular}} & \begin{tabular}[c]{@{}c@{}}\textit{f0} not a cue \\ (Type IIa)\end{tabular} & \begin{tabular}[c]{@{}c@{}}\textit{f0} is a cue \\ (Type IIb)\end{tabular} & \\
    \hline
    \textbf{Tone contrast} & \begin{tabular}[c]{@{}c@{}}VQ not a cue \\ (Type IIIa)\end{tabular} & \begin{tabular}[c]{@{}c@{}}VQ is a cue \\ (Type IIIb)\end{tabular} & \begin{tabular}[c]{@{}c@{}} Orthogonal \\ (Type IVa)\end{tabular} & \begin{tabular}[c]{@{}c@{}} Fused \\ (Type IVb)\end{tabular} & \begin{tabular}[c]{@{}c@{}} Mixed \\ (Type IVc)\end{tabular} \\
    \lspbottomrule
    \end{tabular}
\end{table}