%-------------------------------------------------------------
\chapter{Introduction} \label{chap:introduction}
%-------------------------------------------------------------


%-------------------------------------------------------------
\section{What is Voice Quality} \label{sec:voice_quality}
%-------------------------------------------------------------
Voice quality refers to the long-term characteristics of an individual's voice \citep{abercrombieElementsGeneralPhonetics1967,laverPhoneticDescriptionVoice1980}. However, the term voice quality is often used more narrowly to refer to the how the larynx affects the phonetic characteristics of speech sounds. When the larynx is manipulated during speech production, it results in qualities of voice that we describe as being modal, breathy, and creaky. Modal voice is the most common type of phonation and is characterized by a regular vibration of the vocal folds, resulting in a clear and full sound. Breathy voice, on the other hand, is characterized by a partial closure of the vocal folds, allowing more air to escape during voicing, resulting in a breathy or airy quality. Creaky voice is characterized by a tight closure of the vocal folds during voicing, resulting in irregular vibration.

Additionally, the term phonation is often used interchangeably with voice quality to refer to the same phenomenon (e.g., \cite{keatingPhonationContrastsLanguages}). However, there are some contexts in which these terms are used with slightly different meanings. Phonation is often used in a more technical sense to refer specifically to the production of sound by the vocal folds, while voice quality can encompass a broader range of characteristics, including resonance and articulation \citep{eslingVoiceQualityLaryngeal2019}. Another way these two terms are used is to make a phonetics (i.e., voice quality) versus phonology distinction (i.e., phonation); \citet{barzilaiContextdependentPhoneticEnhancement2021} uses this contrast in their work on phonation on San Pablo Macuiltianguis Zapotec. 

Languages also make use of voice quality differences to convey paralinguistic information by ``indexing the biological, psychological, and social characteristics of the speaker" \citep{laverVoiceQualityIndexical1968,podesvaStanceWindowLanguageRace2016} or use it for phonemic contrasts \citep{ladefogedSoundsWorldsLanguages1996}. A an example of the former, is that female speakers of American English are often described as having a breathier voice quality than males \citep[e.g.,][]{klattAnalysisSynthesisPerception1990}. In the latter, most Oto-Manguean languages make use of voice quality to distinguish between phonemic contrasts \citep{lillehaugenOtomangueanLanguages2019}. 


%-------------------------------------------------------------
\section{Measuring voice quality} \label{sec:measuring_voice_quality}
%-------------------------------------------------------------

Acoustic measurements are the most common way of measuring phonation and involve analyzing the sound waves produced during speech (see \cite{garellekPhoneticsVoice2019} for a detailed discussion). The oldest and most common way this is accomplished is with spectral silt measures which reflect the open quotient, which is the proportion of the glottal cycle during which the glottis is open \citep{holmbergComparisonsAerodynamicElectroglottographic1995}. These measures have their origin in \posscitet{fischer-jorgensenPhoneticAnalysisBreathy1968} investigation into breathy vowels in Gujarati. She observed that the amplitude of the first harmonic (H1) has higher for vowels labeled as breathy/murmured than for vowels that were plain/modal. As a way to normalize across the signal and mitigate the effects of the high-pass filter, she subtracted the amplitude of the second harmonic (H2) from the amplitude of the first harmonic (H1) to create a measure that is now commonly referred to as H1$-$H2. 

Subsequent research has shown that this measure is effective at distinguishing not only breathy and modal phonation but creaky phonation as well in a variety of languages (see \cite{garellekTheoreticalAchievementsPhonetics2022} for a history). This has led to the widespread adoption of this measure and for some researchers to claim that ``H1$-$H2 may be a (near-)universal acoustic measure of phonation \citep[8]{espositoCrosslinguisticPatternsPhonation2020}". However, research has also shown that this measure is not always effective at distinguishing between phonation types in all languages (e.g., \cite{brinkerhoffUsingResidualH12025,chaiH1H2AcousticMeasure2022,espositoVariationContrastivePhonation2010,simpsonFirstSecondHarmonics2012}). 

Based on a multiplicity of research into voice quality showing that voice quality is multidimensional, \citet{kreimanUnifiedTheoryVoice2014,kreimanValidatingPsychoacousticModel2021} have proposed a unified model that incorporates several acoustic measures that capture the spectral slope, the inharmonic source excitation, the time-varying source characteristics, and the vocal tract transfer function. They have that voice quality can best be understood as a combination of these different acoustic measures and that no single measure is sufficient to capture the complexity of voice quality.



% \begin{itemize}
%     \item The most common way that this is done is with spectral silt measures which reflect the open quotient.
%     \begin{itemize}
%         \item . 
%     \end{itemize} 
%     \item Other measures that reflect the periodicity of the signal are also consulted such as HNR and CPP. Typically nonmodal phonation is also aperiodic and will have a lower score compared to the modal. 
%     \item They also state that ``H1-H2 may be a (near-)universal acoustic measure of phonation'' (8).
%     \item They further say that this measure works the best the most often but there are several exceptions where other H1-X measures are more robust at capturing the phonation contrasts in the language. 
%     \begin{itemize}
%         \item As discussed in \citet{chaiH1H2AcousticMeasure2022}, what they are actually trying to do is get at the amplitude differences in H1 as first observed by \citet{fischer-jorgensenPhoneticAnalysisBreathy1968,laverVoiceQualityIndexical1968}.
%         \item I think this is where, I and other researchers fell into a fallicy where we think that just the spectral-tilt is what matters instead of realizing what those measures where trying to accomplish which was to normalize H1 for comparison. 
%     \end{itemize}
%     \item \citet{espositoCrosslinguisticPatternsPhonation2020} further discuss the role that EGG play. 
%     \item Two things that I think are relavent for my dissertation is the discussion around localization of non-modal phonation and how phonation relates to tone. 
%     \item For the discussion around localization, they keep in line with \citet{silvermanLaryngealComplexityOtomanguean1997} who says that phonation is located in certain portions of the vowel. 
%     \item Localization of non-modal phonation is also important in languages with complex ``clusters'' (i.e., phonetic sequences) of phonation type, which involve at least one non-modal phonation localized to a portion of the vowel. 
%     \begin{itemize}
%         \item !Xóõ distinguishes clusters of breathy-creaky, pharyngealized-creaky, and breathy-pharyngealized phonation in addition to modal, breathy, creaky, and pharyngealized voice
%     \end{itemize}
%     \item In terms of tone and phonation interactions, \citet{espositoCrosslinguisticPatternsPhonation2020} claim that there are four different types of interactions based on whether or not tone and phonation is contrastive in the language. 
%     \item Table~\ref{tab:Typology} shows each of the different interactions between tone and phonation. In this table each row represents whether or not tone/\textit{f0}/pitch accent is contrastive and each column represents whether or not phonation is contrastive. 
%     \item Each cell represent each of the four different types of languages (I-IV). 
%     \item Types II-IV are further subdivided based on whether or not tone and phonation are cues for each other. 
% \end{itemize}

%-------------------------------------------------------------
\section{Interactions between Voice Quality and Tone} \label{sec:interactions_between_voice_quality_and_tone}
%-------------------------------------------------------------

Typologically speaking voice quality can interact with tone in a variety of ways. In \posscitet{espositoCrosslinguisticPatternsPhonation2020} typology of phonation and tone, they describe four different types of interactions between tone and phonation based on whether or not tone and phonation are contrastive in the language. These interactions are summarized in Table~\ref{tab:typology}, where each row indicates whether or not tone is contrastive in the language and each column indicates whether phonation is contrastive or not. Beginning in the top left cell, the first type of language is one where neither tone nor phonation are contrastive (Type I). The second type of language is one where phonation is contrastive but tone is not (Type II). The third type of language is one where tone is contrastive but phonation is not (Type III). Finally, the fourth type of language is one where both tone and phonation are contrastive (Type IV). For types II-IV, the cells are further subdivided based on whether or not tone and phonation are cues for each other. Each of these types of languages are discussed in more detail below.

\begin{table}[h!]
    \centering
    \caption{\posscitet{espositoCrosslinguisticPatternsPhonation2020} language types based on the contrastiveness of tone and phonation.}
    \label{tab:typology}
    \begin{tabular}{c|c|c|c|c}
        \lsptoprule
     & \multicolumn{2}{c|}{\textbf{No phonation contrast}} & \multicolumn{2}{c}{\textbf{Phonation contrast}} \\
     \hline
    \textbf{No tone contrast} & \multicolumn{2}{c|}{\begin{tabular}[c]{@{}c@{}}No contrasts \\ (Type I)\end{tabular}} & \begin{tabular}[c]{@{}c@{}}\textit{f0} not a cue \\ (Type IIa)\end{tabular} & \begin{tabular}[c]{@{}c@{}}\textit{f0} is a cue \\ (Type IIb)\end{tabular} \\
    \hline
    \textbf{Tone contrast} & \begin{tabular}[c]{@{}c@{}}VQ not a cue \\ (Type IIIa)\end{tabular} & \begin{tabular}[c]{@{}c@{}}VQ is a cue \\ (Type IIIb)\end{tabular} & \begin{tabular}[c]{@{}c@{}} Orthogonal \\ (Type IVa)\end{tabular} & \begin{tabular}[c]{@{}c@{}} Fused \\ (Type IVb)\end{tabular} \\
    \lspbottomrule
    \end{tabular}
\end{table}

Type I languages are those languages that lack both tonal contrasts and phonation contrasts. According to \citet{espositoCrosslinguisticPatternsPhonation2020}, type I languages include most Australian Aboriginal languages, most Austronesian languages, most Indo-European languages, most Afro-Asiatic languages, Standard Khmer, and the Turkic languages. Instead tone and phonation are used for paralinguistic purposes such as ``indexing the biological, psychological, and social characteristics of the speaker \citep{laverVoiceQualityIndexical1968}" or racial characteristics \citep{podesvaStanceWindowLanguageRace2016}. Additionally, these languages may use voice quality for pragmatic purposes such as signaling emphasis or emotion.

Type II languages are similar to Type I languages in that they lack tonal contrasts, but instead have phonation contrasts on the vowel; however, these phonation contrasts are further subdivided into two subtypes (IIa and IIb) based on whether or not changes in the fundamental frequency (\textit{f0}) are a cue to the phonation. Type IIa languages are those where \textit{f0} is not a cue. According to \citet{espositoCrosslinguisticPatternsPhonation2020}, these languages are quite rare and only two languages are known to exhibit this behavior: Danish\footnote{\citet{frazierPhoneticsYucatecMaya2013,penaStodTimingDomain2022,penaProductionPerceptionStod2024} contest this claim and show rather convincingly that stød actually has \textit{f}0 cues.} \citep{gronnumDanishStodLaryngealization2013} and Gujarati \citep{khanPhoneticsContrastivePhonation2012}. Type IIb languages are those where \textit{f0} functions as a cue for the different phonation types. These languages are sometimes called ``register languages'' and are frequently found in Southeast Asian languages such as Chanthaburi Khmer, Chong, Javanese, Kedang, Mon, Suai, and Wa \citep[e.g.,][]{brunelleTonePhonationSoutheast2016,dicanioPhoneticsRegisterTakhian2009,samelyKedangEasternIndonesia1991,waylandAcousticCorrelatesBreathy2003}. For type IIb languages the way in which \textit{f0} is a cue is language specific. For example, breathy vowels in Wa and Kedang both begin with a lower \textit{f0} than their clear/modal counterparts \citep{samelyKedangEasternIndonesia1991}, while breathy vowels in Chanthaburi Khmer have a higher \textit{f0} than their clear counterparts \citep{waylandAcousticCorrelatesBreathy2003}. 

Type III languages are the mirror image of Type II languages in that they have a contrast in tone but lack a phonation contrast. These languages are further subdivided into two subtypes based on whether phonation is a cue for the tonal contrasts (Type IIIa vs. Type IIIb). Type IIIa languages are those where tone is contrastive and phonation is not a cue for those tonal contrasts. According to \citep{espositoCrosslinguisticPatternsPhonation2020}, this includes Japanese, Navajo, Punjabi, Manange, many West African languages, Swedish, and Central Thai. Type IIIb languages, on the other hand, have tone and use phonation as a cue for one or more tones. For example, Mandarin Chinese's falling-rising tone is frequently accompanied with creak \citep[e.g.,][]{kuangCovariationVoiceQuality2017}.  According to \citep{espositoCrosslinguisticPatternsPhonation2020}, other examples of this include Cantonese Yue, Khmu' Rawk, Mandarin, Pakphanang Thai, Phnom Penh Khmer, and Yueyang Xiang.

The last type of combination of tone and phonation is Type IV languages. These languages have contrastive tone and phonation and are also subdivided into two subtypes (IVa and IVb). Type IVa languages are those where tone and phonation are allowed to combine freely with no restrictions; \citet{espositoCrosslinguisticPatternsPhonation2020} call these languages \textit{orthogonal} and are exemplified by languages such as Dinka, Mazatec, Mpi, Yalálag Zapotec, and Yi languages (see \cite{espositoCrosslinguisticPatternsPhonation2020} for references).

In comparison, Type IVb languages also make use of contrastive tone and phonation, but they are fused in such a way that it is impossible to state if they have tonal contrasts or phonation contrasts. \citet{espositoCrosslinguisticPatternsPhonation2020} call these languages \textit{fused} but are also commonly called ``register languages'' and are primarily found in Southeast Asia \citep{brunelleTonePhonationSoutheast2016,enfieldArealLinguisticsMainland2005,masicaDefiningLinguisticArea1976}. Furthermore, \citet{espositoCrosslinguisticPatternsPhonation2020} also claim that most Zapotec languages also fall into this type of language because there are specific tones that only arise with specific phonation types. For example, Santa Ana del Valle Zapotec's rising tone can only occur with modal phonation and its falling tone can only occur with breathy phonation \citep{espositoVariationContrastivePhonation2010}; additionally, Isthmus Zapotec is notable for its high tone and falling tone being unable to appear with any phonation in monosyllables \citep{pickettIsthmusJuchitanZapotec2010}. 

Type IV languages are also collectively referred to as \textit{laryngeally complex} languages by \citet{silvermanLaryngealComplexityOtomanguean1997,silvermanPhasingRecoverability1997} owing to the fact that tone and phonation are both produced in the larynx and the interactions between the two are complex. The complexity in these languages arises from the variety of ways that tone and phonation are produced. For \citeauthor{silvermanLaryngealComplexityOtomanguean1997}, laryngeal complexity most frequently manifests as a form of phasing between tone and phonation. Phasing is the idea that the most optimal way for tone and phonation to be produced sequentially in a vowel. This often manifests as a portion of the vowel being produced with nonmodal phonation and another portion of the vowel being produced with modal phonation. The rationale behind this is because tone is easily perceived with modal phonation, while nonmodal phonation makes it difficult to perceive tone. However, \citeauthor{silvermanLaryngealComplexityOtomanguean1997} also states that in laryngeally complex languages, tone and phonation can also be produced simultaneously if the nonmodal phonation is produced more weakly; this is argued to be the distinction between the laryngeally complex languages of Jalapa Mazatec and Mpi \citep{ladefogedSoundsWorldsLanguages1996,silvermanLaryngealComplexityOtomanguean1997}.  

%------------------------------------
\section{Research question} \label{sec:research_question}
%------------------------------------

This dissertation investigates the acoustics of voice quality in Santiago Laxopa Zapotec and its interactions with tone. Specifically, it aims to answer the following questions: (i) does the recently proposed residual H1* acoustic measure more effective captures the phonation contrasts between Santiago Laxopa Zapotec's four phonations than the traditional H1*$-$H2* acoustic measure; (ii) how is the acoustic landscape of voice quality in Santiago Laxopa Zapotec structured; (iii) which acoustic measures most effectively capture and classify the voice quality contrasts; (iv) and how do those acoustic measures help explain Santiago Laxopa Zapotec's laryngeal complexity? 

These four questions form a cohesive program of study and are important for several reasons. First, if new acoustic measures are proposed we need to determine to what extent they are effective at capturing the phonation contrasts in a language when compared to more established measures. If the new measure is shown to be more effective than a more established acoustic measure then it follows that the new measure should be considered when performing our acoustic investigations. Second, understanding the acoustic landscape of voice quality in a language can provide insights into how voice quality is structured. Third, identifying the most effective acoustic measures for classifying voice quality contrasts can help researchers develop more accurate and reliable methods for analyzing voice quality. Fourth, investigating the interactions between tone and voice quality can shed light on the complexities of these interactions and their implications for our understanding of phonetics and phonology. Finally, answering these questions with respect to Santiago Laxopa Zapotec, which is a language that has not been extensively studied, can contribute to our theoretical understanding by allowing us to test claims and verify the robustness of the proposed measures and theories across languages.

%------------------------------------
\section{Outline of the dissertation } \label{sec:conclusion}
%------------------------------------

The rest of this dissertation is organized as follows. Chapter~\ref{ch:SLZ} provides a detailed description of the vowel, voice quality, and tone system of Santiago Laxopa Zapotec, an Oto-Manguean language spoken in Oaxaca, Mexico. This chapter includes a description of the phonetic and phonological properties of the vowels, the phonation types, and the tonal contrasts in the language. Chapter~\ref{ch:residual_h1} presents the results of an acoustic analysis of the voice quality contrasts in SLZ, focusing on the recently proposed residual H1* acoustic measure from \citet{chaiH1H2AcousticMeasure2022} and its effectiveness its effectiveness in distinguishing between the phonation types while comparing its results to the more traditional H1*$-$H2* acoustic measure. The results show that this measure is more effective than the traditional H1*$-$H2* measure in distinguishing the phonation types in Santiago Laxopa Zapotec adding credence to this acoustic measure and its adoption by researchers. 

Chapter~\ref{ch:acousticlandscape} presents the results of an acoustic analysis investigating the acoustic landscape that Santiago Laxopa's voice quality occupies using multidimensional scaling. This chapter demonstrates that voice quality in Santiago Laxopa Zapotec occupies a 3-dimensional space where the first and third dimensions correlate with spectral slope while the second dimension correlates with the periodicity and noise of the signal. This chapter also shows that Santaigo Laxopa Zapotec's 3-dimensional space is consistent with \posscitet{keatingCrosslanguageAcousticSpace2023} findings that cross-linguistically voice quality occupies a primarily 2 dimensional space that is also defined by spectral slope and periodicity/noise. These findings suggest that the acoustic landscape of voice quality is not only consistent across languages but also that the dimensions of this space are defined by similar acoustic properties. Furthermore, the findings also show that residual H1* is highly correlated with the spectral slope dimensions of the acoustic landscape, which further suggests that this measure is effective at capturing the spectral properties of voice quality and important for defining the acoustic space.

Chapter~\ref{ch:revealing_trees} presents a random forest analysis about which acoustic measures are most effective at distinguishing the phonation types in Santiago Laxopa Zapotec. This chapter shows that the most effective acoustic measures for distinguishing the voice quality in Santiago Laxopa Zapotec are very similar to those that are correlated with the dimensions of the acoustic landscape in Chapter~\ref{ch:acousticlandscape} with the addition of duration. 

Based on the findings in Chapters~\ref{ch:acousticlandscape} and \ref{ch:revealing_trees}, Chapter~\ref{ch:testing_lc} discusses the implications of these findings for our understanding of voice quality and its interactions with tone. This chapter specifically investigates \posscitet{silvermanLaryngealComplexityOtomanguean1997} claims about how tone and voice quality must be phased. To investigate these claims about phasing three generalized additive mixed models were assessed on the acoustic measures of \textit{f}0, Strength of Excitation, and HNR $<$ 1500 Hz. The findings show that despite \posscitet{herrerazendejasAmuzgoZapotecTwo2000} claims that Zapotec languages do not show phasing between tone and voice quality there is evidence that suggests that there is a phasing between tone and voice quality in Santiago Laxopa Zapotec. Specifically that the breathy and checked vowels are associated with a modal portion followed by a nonmodal portion of the vowel, being consistent with \posscitet{silvermanLaryngealComplexityOtomanguean1997} postvocalic phasing pattern. Rearticulated vowels, on the other hand, are associated with a nonmodal portion followed by a modal portion of the vowel, being consistent with \posscitet{silvermanLaryngealComplexityOtomanguean1997} prevocalic phasing pattern. These findings suggest that the interactions between tone and voice quality in Santiago Laxopa Zapotec are more complex than previously thought and that phasing plays a significant role in these interactions. Furthermore, the findings suggest that the implicational hierarchy for the interactions between the phasing relationships between tone and voice quality proposed by \citet{silvermanLaryngealComplexityOtomanguean1997} does hold true for Santiago Laxopa Zapotec. 

Finally, Chapter~\ref{ch:conclusion} concludes the dissertation by summarizing the main findings and contributions of this research to our understanding of voice quality. It also discusses potential avenues for future research on voice quality and its interactions with tone and highlights the importance of continued research in this area to further our understanding of the complexities of voice quality and its role in language.

% \begin{table}[h!]
%     \centering
%     \caption{Updated language types based on the lexically contrastive nature of \textit{f0}/tone/pitch accents (rows) and voice quality on vowels (columns) and their interactions}
%     \label{tab:TypologyUpdated}
%     \begin{tabular}{c|c|c|c|c|c}
%         \lsptoprule
%      & \multicolumn{2}{c|}{\textbf{No VQ contrast}} & \multicolumn{3}{c}{\textbf{VQ contrast}} \\
%      \hline
%     \textbf{No tone contrast} & \multicolumn{2}{c|}{\begin{tabular}[c]{@{}c@{}}No contrasts \\ (Type I)\end{tabular}} & \begin{tabular}[c]{@{}c@{}}\textit{f0} not a cue \\ (Type IIa)\end{tabular} & \begin{tabular}[c]{@{}c@{}}\textit{f0} is a cue \\ (Type IIb)\end{tabular} & \\
%     \hline
%     \textbf{Tone contrast} & \begin{tabular}[c]{@{}c@{}}VQ not a cue \\ (Type IIIa)\end{tabular} & \begin{tabular}[c]{@{}c@{}}VQ is a cue \\ (Type IIIb)\end{tabular} & \begin{tabular}[c]{@{}c@{}} Orthogonal \\ (Type IVa)\end{tabular} & \begin{tabular}[c]{@{}c@{}} Fused \\ (Type IVb)\end{tabular} & \begin{tabular}[c]{@{}c@{}} Mixed \\ (Type IVc)\end{tabular} \\
%     \lspbottomrule
%     \end{tabular}
% \end{table}