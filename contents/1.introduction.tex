%--------------------------------------------------------------------------
%  Introduction
%--------------------------------------------------------------------------

%--------------------------------------------------------------------------
\chapter{Introduction} \label{chap:introduction}
%--------------------------------------------------------------------------


%--------------------------------------------------------------------------
\section{What is Voice Quality} \label{sec:voice_quality}
%--------------------------------------------------------------------------
Voice quality describes the state of the larynx during phonation, when the vocal folds are set in motion. Languages make use of voice quality for paralinguistic purposes, such as conveying indexation of ``biological, psychological, and social characteristics of the speaker'' \citep{laverVoiceQualityIndexical1968} and racial identity \citep{podesvaStanceWindowLanguageRace2016}. 

Voice quality is also used linguistically. In English, it is often the case that we use creaky voice to indicate that we are at the end of an utterance \citep[e.g.,][]{garellekProductionPerceptionGlottal2013}. In many other languages, voice quality is used as part of the phonological system. Most famously, Gujarati has a phonemic contrast between breathy and modal voice in vowels \citep[e.g.,][]{fischer-jorgensenPhoneticAnalysisBreathy1968,espositoContrastiveBreathinessConsonants2012, khanPhoneticsContrastivePhonation2012,espositoDistinguishingBreathyConsonants2019}. 

\citet{espositoCrossLinguisticPatterns2020}

%--------------------------------------------------------------------------
\section{Voice Quality and Tone} \label{sec:voice_quality_and_tone}
%--------------------------------------------------------------------------

%--------------------------------------------------------------------------
\section{Voice Quality and Phonation} \label{sec:voice_quality_and_phonation}
%--------------------------------------------------------------------------

% In other languages, voice quality can be used to distinguish between phonemes, such as in Jalapa Mazatec, where voice quality is used to distinguish between voiced and voiceless stops \citep{merrillVoiceQualityJalapa2011}. Voice quality is also used to convey information about the speaker's emotional state, such as in the case of breathiness, which is often associated with sadness \citep{hentonBreathinessVoiceQuality1996}.
